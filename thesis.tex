\documentclass[pdftex,cyrillic,14pt,a4page,book]{extreport}
\usepackage[bulgarian]{babel}

\usepackage[margin=2cm]{geometry}% http://ctan.org/pkg/geometry
\usepackage[pdftex]{graphicx}

\usepackage{./titlesec/titlesec}


  
\titleformat{\chapter}%
  {\normalfont\bfseries\Huge}{\thechapter.}{10pt}{}

\usepackage{afterpage}
\newcommand\blankpage{%
    \null
    \thispagestyle{empty}%
    \newpage}


\begin{document}
\begin{titlepage}
	\begin{center}
	\includegraphics[scale=1.2]{./NBU_logo.jpg}\\[0.3cm]
    \textbf{\Large НОВ БЪЛГАРСКИ УНИВЕРСИТЕТ\\[0.4cm]}
    \textbf{\Large Департамент Информатика\\[0.4cm]}
    \textbf{\Large Бакалавърка програма Информатика\\[3cm]}
   
		\textbf{\LARGE Автоматизиран биоинформатичен анализ на генетични варианти, потенциално свързани със стареенето\\[2cm]}
		\begin{Large}
		Дипломна работа на\\[0.2cm]
		Михаил М. Здравков\\[3cm]
		\end{Large}
		\begin{minipage}{0.48\textwidth}
			\begin{flushleft} \large
				\emph{Научни ръководители:} \\
				доц. д-р Милена Георгиева \\
				Момчил Топалов
			\end{flushleft}
		\end{minipage}
			\begin{minipage}{0.48\textwidth}
			\begin{flushright} \large
				\emph{Ръководител катедра:} \\
				гл. ас. д-р Методи Трайков\\
				\clearpage
			\end{flushright}
		\end{minipage}

		\vfill

		% Bottom of the page
		{\large София 2022}

	\end{center}
\end{titlepage}


\afterpage{\blankpage}

% Include title page and blank page in page numbering
\addtocounter{page}{1}

\tableofcontents
\pagebreak
\afterpage{\blankpage}


\newgeometry{
	left=30mm,
    right=20mm,
    top=20mm,
    bottom=20mm}


\chapter{Увод}
%\addtocounter{chapter}{1}
%\addcontentsline{toc}{chapter}{Увод}
\paragraph{}

Стареенето е естествен процес, който има огромно значение както за отделния индивид, така и за обществото като цяло. С напредването на възрастта, рискът от разнообразни заболявания като рак, болест на Алцхаймер, диабет, сърдечно-съдови заболявания и др. нараства значително. Смята се, че около две-трети от смъртните случаи при хора се дължат на заболявания, свързани с възрастта. Същевременно, с глобалното нарастване на средната продължителност на живота, проблемите на стареенето засягат все повече хора и имат все по-голямо обществено значение. От социална гледна точка, стареенето оказва значителен икономически и демографски ефект.

\paragraph{}
Установено е, че процесът на стареене се влияе както от генетични, така и от епигенетични фактори. Въпреки това, този процес все още не е достатъчно добре разбран от науката, поради което е трудно да се създадат ефективни методи за терапия и справяне с негативните му ефекти.

\paragraph{}
Основен подход при изследването на генетичната основа на стареенето е анализът на генетични варианти. При такива изследвания е необходима обработката на големи обеми от данни, което налага нуждата от използване на специализиран биоинформатичен софтуер. Налични са множество различни инструменти, покриващи различни аспекти от обработката на файлове с генетични варианти - анотация, филтриране, анализ и тн. Повечето от тях, обаче, изискват значителни технически познания, което ги прави трудни за използване от специалисти в други области, като биология и генетика.

\paragraph{}
Целта на настоящата дипломна работа е създаването на интегрирана софтуерна система за биоинформатични изследвания на генетични варианти и предсказване на тяхната потенциална асоциация с процеса на стареене. Надяваме се, чрез създаване на по-достъпен инструмент, да допринесем за бъдещи изследвания на процеса на стареене и за търсенето на ефективни терапии против негативните му ефекти.
            
\chapter{Литературен обзор}
%\addcontentsline{toc}{chapter}{Литературен обзор}
\section{Значение на стареенето}
\end{document}